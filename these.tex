\documentclass[a4paper, twoside, 12pt]{stylethese}

%%%%%%%%%%%%%%%%%%%%%%%%%%%%%%%%%%%%%%%%%%%%%%%%%%
% Package loading

%% Liens hypertextes: 
\usepackage[linktocpage=true, linkcolor=cyan, citecolor=cyan, colorlinks=true,
urlcolor=blue, pagebackref]{hyperref}
%% Citations: 
\usepackage{natbib}
%% Bibliographie par chapître
\usepackage{chapterbib}
%% Mathématiques:
\usepackage{amssymb,amsmath}
%% Encoding:
\usepackage[utf8]{inputenc}
%% Check overfull box:
\usepackage{txfonts,empheq}
%% Font used:
\usepackage{lmodern}
%% For font warnings in code block:
\usepackage[T1]{fontenc}


\usepackage[english,french]{cleveref}
\usepackage{upgreek}
\usepackage{textcomp}
\usepackage{tabularx}
\usepackage{longtable}
\usepackage{ltxtable}
\usepackage{pdflscape}
\usepackage{booktabs}
\usepackage{afterpage}
\usepackage{floatpag}
\usepackage[babel=true]{csquotes}
\usepackage{pifont}
\usepackage{soulutf8}
\usepackage{caption}
\usepackage{color}
\usepackage{ulem}

\raggedbottom % Évite les espaces trop gands entre chaque section.


%%%%%%%%%%%%%%%%%%%%%%%%%%%%%%%%%%%%%%%%%%%%%%%%%%
% Configuration for code displayed in intro.tex  
% From Kevin
\IfFileExists{upquote.sty}{\usepackage{upquote}}{}
% use microtype if available
\IfFileExists{microtype.sty}{%
\usepackage[]{microtype}
\UseMicrotypeSet[protrusion]{basicmath} % disable protrusion for tt fonts
}{}
\PassOptionsToPackage{hyphens}{url} % url is loaded by hyperref
\urlstyle{same}  % don't use monospace font for urls
\usepackage{fancyvrb}
\newcommand{\VerbBar}{|}
\newcommand{\VERB}{\Verb[commandchars=\\\{\}]}
\DefineVerbatimEnvironment{Highlighting}{Verbatim}{commandchars=\\\{\}}
% Add ',fontsize=\small' for more characters per line
\newenvironment{Shaded}{}{}
\newcommand{\KeywordTok}[1]{\textcolor[rgb]{0.00,0.44,0.13}{\textbf{#1}}}
\newcommand{\DataTypeTok}[1]{\textcolor[rgb]{0.56,0.13,0.00}{#1}}
\newcommand{\DecValTok}[1]{\textcolor[rgb]{0.25,0.63,0.44}{#1}}
\newcommand{\BaseNTok}[1]{\textcolor[rgb]{0.25,0.63,0.44}{#1}}
\newcommand{\FloatTok}[1]{\textcolor[rgb]{0.25,0.63,0.44}{#1}}
\newcommand{\ConstantTok}[1]{\textcolor[rgb]{0.53,0.00,0.00}{#1}}
\newcommand{\CharTok}[1]{\textcolor[rgb]{0.25,0.44,0.63}{#1}}
\newcommand{\SpecialCharTok}[1]{\textcolor[rgb]{0.25,0.44,0.63}{#1}}
\newcommand{\StringTok}[1]{\textcolor[rgb]{0.25,0.44,0.63}{#1}}
\newcommand{\VerbatimStringTok}[1]{\textcolor[rgb]{0.25,0.44,0.63}{#1}}
\newcommand{\SpecialStringTok}[1]{\textcolor[rgb]{0.73,0.40,0.53}{#1}}
\newcommand{\ImportTok}[1]{#1}
\newcommand{\CommentTok}[1]{\textcolor[rgb]{0.38,0.63,0.69}{\textit{#1}}}
\newcommand{\DocumentationTok}[1]{\textcolor[rgb]{0.73,0.13,0.13}{\textit{#1}}}
\newcommand{\AnnotationTok}[1]{\textcolor[rgb]{0.38,0.63,0.69}{\textbf{\textit{#1}}}}
\newcommand{\CommentVarTok}[1]{\textcolor[rgb]{0.38,0.63,0.69}{\textbf{\textit{#1}}}}
\newcommand{\OtherTok}[1]{\textcolor[rgb]{0.00,0.44,0.13}{#1}}
\newcommand{\FunctionTok}[1]{\textcolor[rgb]{0.02,0.16,0.49}{#1}}
\newcommand{\VariableTok}[1]{\textcolor[rgb]{0.10,0.09,0.49}{#1}}
\newcommand{\ControlFlowTok}[1]{\textcolor[rgb]{0.00,0.44,0.13}{\textbf{#1}}}
\newcommand{\OperatorTok}[1]{\textcolor[rgb]{0.40,0.40,0.40}{#1}}
\newcommand{\BuiltInTok}[1]{#1}
\newcommand{\ExtensionTok}[1]{#1}
\newcommand{\PreprocessorTok}[1]{\textcolor[rgb]{0.74,0.48,0.00}{#1}}
\newcommand{\AttributeTok}[1]{\textcolor[rgb]{0.49,0.56,0.16}{#1}}
\newcommand{\RegionMarkerTok}[1]{#1}
\newcommand{\InformationTok}[1]{\textcolor[rgb]{0.38,0.63,0.69}{\textbf{\textit{#1}}}}
\newcommand{\WarningTok}[1]{\textcolor[rgb]{0.38,0.63,0.69}{\textbf{\textit{#1}}}}
\newcommand{\AlertTok}[1]{\textcolor[rgb]{1.00,0.00,0.00}{\textbf{#1}}}
\newcommand{\ErrorTok}[1]{\textcolor[rgb]{1.00,0.00,0.00}{\textbf{#1}}}
\newcommand{\NormalTok}[1]{#1}
\IfFileExists{parskip.sty}{%
\usepackage{parskip}
}{% else
\setlength{\parindent}{0pt}
\setlength{\parskip}{6pt plus 2pt minus 1pt}
}

%%%%%%%%%%%%%%%%%%%%%%%%%%%%%%%%%%%%%%%%%%%%%%%%%%
% Variable definitions
\title{Titre de la thèse}
%\titleEng{}
\author{Prénom NOM}
%\docschool{\'Ecole Doctorale PNC}
%\lab{Un merveilleux laboratoire}
%\field{[Intitulé du diplome]}
%\defensedate{[XX mois année]}

%% Number of supervisors:
\nsupervisors{2}
\addsupervisor{1}{Prénom NOM et}
\addsupervisor{2}{Prénom NOM}

%% Number of jury members:
\njurymembers{2}
%% Définition
\addjurymember{1}{Mme.}{Prénom Nom}{Grade}{Établissement}{Rapportrice}
\addjurymember{2}{M.}{Prénom Nom}{Grade}{Établissement}{Directeur de thèse}

% These sur articles (voir stylethese.cls)
\Articles

\begin{document}
\pdfstringdefDisableCommands{
% Alain: I do not know what it is exactly but liminaires will not compile
% without it 
\let\MakeUppercase\relax}


% ----------------------------------------------------------------------%
% 2- Liminaires de la thèse.                                            %
% ----------------------------------------------------------------------%

\include{liminaires}


% ----------------------------------------------------------------------%
% 3- Corps de la thèse.                                                %
% ----------------------------------------------------------------------%

\debutcorps
\cleardoublepage
% Utiliser      chapitres.tex pour une thèse (ou mémoire) traditionnelle.
% Remplacer par articles.tex  pour une thèse (ou mémoire) par articles.


\introduction
\selectlanguage{french}

\input{intro.tex}
\cleardoublepage

\selectlanguage{english}

\input{./chapitre1/chap1.tex}
\input{./chapitre2/chap2.tex}
\input{./chapitre3/chap3.tex}

\cleardoublepage


\conclusion
\selectlanguage{french}
\input{conclu.tex}
\cleardoublepage

% ----------------------------------------------------------------------%
% 4 - Appendices de la thèse.                                           %
% ----------------------------------------------------------------------%

\input{annexe1/annexe1.tex}
\input{annexe2/annexe2.tex}

% ----------------------------------------------------------------------%
% 5 - Bibliographie.                                                    %
% ----------------------------------------------------------------------%

\selectlanguage{english}

\begin{singlespace}
  \makeatletter
  \phantomsection\addcontentsline{toc}{chapter}{\MakeUppercase{\@references}}
  \makeatother
  \bibliographystyle{apalike}
  \bibliography{mybiblio} % Ici mettre le nom de la biblio, ici mylib.bib
\end{singlespace}

% \begin{singlespace}
%   \makeatletter
%   \phantomsection\addcontentsline{toc}{chapter}{\MakeUppercase{\@references}}
%   \makeatother
%   \selectlanguage{english}
%   % \bibliographystyle{elsevier-harvard.csl} % Ici éditer le style
%   \bibliography{/Users/KevCaz/Documents/library.bib} % Ici mettre le nom de la biblio, ici mylib.bib
% \end{singlespace}



% ----------------------------------------------------------------------%
% Fin du document.                                                     %
% ----------------------------------------------------------------------%

\end{document}
